%=== Chapter One ===
\chapter{Introduction}

\begin{proposition}
No set can be both simple and creative.
\end{proposition}

\begin{proof}
The complement of a simple set does not contain an infinite c.e. set. The complement of a creative set does contain an infinite c.e. set. Hence no set can be simple and creative.
\end{proof}

\begin{lemma}\label{lemma_creative}
If $f : \mathbb{N} \to \mathbb{N}$ is computable and injective, then $C_f = \{f(x) : \varphi_x(f(x))\downarrow \}$ is creative.
\end{lemma}
\begin{proof}
Clear.
\end{proof}

\begin{proposition}
Let $A \in \Sigma_1$ and $\#A = \#\mathbb{N}$. Then there exists a creative set $C$ such that $C \subseteq A$.
\end{proposition}

\begin{proof}
$A \in \Sigma_1$ and $\#A = \#\mathbb{N} \implies A=\{a_0, a_1, ... \}$ and $f:i \mapsto a_i$ is injective with $rng(f)=A$. 
Hence $C=\{f(i) | \varphi_i(f(i))\downarrow, i \in \mathbb{N}\}$ is creative by \autoref{lemma_creative} and $C \subseteq	 rng(f) = A$.
\end{proof}

\section{The structure of the Turing degrees $\mathcal{D}$}

\begin{defn}For an ordinal $\alpha \in \mathtt{Ord}$ define $\emptyset^{(\alpha)}$ as follows:\\
i) if $\alpha=0$, then $\emptyset^{(0)}:=\emptyset$,\\
ii) if $\alpha=s(\beta)$ is a successor ordinal with a predecessor $\beta$, then $\emptyset^{s(\beta)}:=\{x | \phi^{\emptyset^{(\beta)}}_x(x) \downarrow \}$,\\
iii) if $\alpha$ is a limit ordinal ($\forall \beta \in \mathtt{Ord}. s(\beta) \not = \alpha$), then $\emptyset^{(\alpha)}=sup(\cup_{\beta < \alpha} \emptyset^{(\beta)})$.
\end{defn}

Let $\alpha$ be a limit ordinal. If $\alpha < \omega_1^{CK}$ then $\emptyset^{(\alpha)}:=sup(\cup_{\beta < \alpha} \emptyset^{(\beta)})=\{\langle x, \beta \rangle \in \mathbb{N} : x \in \emptyset^{(\beta)} \land \beta < \alpha\}$. In other case we do not know whether $\emptyset^{(\alpha)}$ is well-defined.

\begin{proposition}
$\exists \alpha \in \mathtt{Ord}. \forall \beta \in \mathtt{Ord}. \alpha < \beta \implies \emptyset^{(\alpha)} \equiv_T \emptyset^{(\beta)} \land \exists A \in \powerset{\mathbb{N}}. A >_T \emptyset^{(\alpha)}$.
\end{proposition}

\begin{proof}
From the definition of $\emptyset^{(\alpha)}$ it follows that if $\beta \le \alpha$ then $\emptyset^{(\beta)} \le_T \emptyset^{(\alpha)}$. Since there are more than $\omega_1$ ordinals, but only $\omega_1$ Turing degrees, the sequence $\emptyset^{(\alpha)}$ has to stabilize, so $\forall \beta \in \mathtt{Ord}. \alpha < \beta \implies \emptyset^{(\alpha)} \equiv_T \emptyset^{(\beta)}$. But there are only countably many oracle machines with the oracle $\emptyset^{(\alpha)}$, therefore there has to be some set $A \in \powerset{\mathbb{N}}$ that cannot be solved even with the oracle $\emptyset^{(\alpha)}$.
\end{proof}