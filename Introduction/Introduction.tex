%=== Chapter One ===
\chapter{Introduction}

\begin{proposition}
No set can be both simple and creative.
\end{proposition}

\begin{proof}
The complement of a simple set does not contain an infinite c.e. set. The complement of a creative set does contain an infinite c.e. set. Hence no set can be simple and creative.
\end{proof}

\begin{lemma}\label{lemma_creative}
If $f : \mathbb{N} \to \mathbb{N}$ is computable and injective, then $C_f = \{f(x) : \varphi_x(f(x))\downarrow \}$ is creative.
\end{lemma}
\begin{proof}
Clear.
\end{proof}

\begin{proposition}
Let $A \in \Sigma_1$ and $\#A = \#\mathbb{N}$. Then there exists a creative set $C$ such that $C \subseteq A$.
\end{proposition}

\begin{proof}
$A \in \Sigma_1$ and $\#A = \#\mathbb{N} \implies A=\{a_0, a_1, ... \}$ and $f:i \mapsto a_i$ is injective with $rng(f)=A$. 
Hence $C=\{f(i) | \varphi_x(f(i))\downarrow, i \in \mathbb{N}\}$ is creative by \autoref{lemma_creative} and $C \subseteq	 rng(f) = A$.
\end{proof}

\begin{lemma}
There exists an infinite set $A$ such that all its infinite subsets are not c.e.
\end{lemma}
\begin{proof}
Define $A=\{p_1^{\chi_K+1}\times...\times p_n^{\chi_K+1} \in \mathbb{N} : n \in \mathbb{N}\}$ where $p_1$, ..., $p_n$ are the first $n$ prime numbers.
If there was a set $B \subseteq A$ then we could enumerate an element $p_1^{\chi_K+1}\times...\times p_n^{\chi_K+1}$ for an arbitrarily large $n$ from which we could decide $K$ for the first $n$ entries, as $n$ arbitrary hence for all entries, but $K$ not decidable.
\end{proof}




